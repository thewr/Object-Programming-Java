L.1
\begin{enumerate}[label=(\alph*),align=left, wide, labelwidth=!, labelindent=0pt]

\item Write a class implementing the Functor<Integer,String> interface, called LengthFun.  It returns the length of a string.
\begin{lstlisting}[language = Java , frame = trBL , firstnumber = last , escapeinside={(*@}{@*)}]
interface functor<R,T> {
   // public R functor(R r, T t);    

    public R apply(T param);
    
}


public class LengthFun implements functor<Integer, String> {    

    private String s;
    
    public void setString(String _s){
        this.s = _s;
    }
    
    public String getString(){
        return s;
    }
    
    public Integer apply(){
        return s.length();
    }
    
    @Override
    public Integer apply(String param) {
        return param.length();
    }    
    
    
    public static void main(String[] args)
    {
        LengthFun lf = new LengthFun();
        String sinput = "Enjoyment";
        lf.setString(sinput);
        System.out.println("LengthFun apply() implemented");
        System.out.println("Length of the word: " + sinput + "; is " + lf.apply());
        System.out.println("LengthFun implementing functor method ... no initializiation of private string necessary");
        System.out.println("Length = " + lf.apply(sinput));
    }
}
\end{lstlisting}  		

\item Define a subclass of $LinkedList<T>$, called $MyList<T>$, that has an additionalgeneric function that "maps" the elements from the list to a new $MyList<R>$ objectthrough a functor object.

\begin{lstlisting}[language = Java , frame = trBL , firstnumber = last , escapeinside={(*@}{@*)}]
import java.util.LinkedList;

interface Functor<R,T> {
    public R apply(T param);
}

class TimesTwoFun implements Functor<Integer,Integer>
{
    @Override
    public Integer apply(Integer param){
        return 2*param;
    }
}

public class MyList<T> extends LinkedList<T>{
    
    public <R> MyList<R> map(Functor<R, Integer> fo) {
        MyList<R> other = new MyList<>();
        for(int i = 0; i < this.size(); i++){
            other.add(fo.apply((Integer) this.get(i)));
        }
        return other;
    }
    
    public static void main(String[] args)
    {
        MyList<Integer> mylist = new MyList();
        TimesTwoFun tt = new TimesTwoFun();
        mylist.add(-2);
        mylist.add(5);
        mylist.add(-1);
        System.out.println("Original list: " + mylist.toString());
        MyList<Integer> mylist2 = mylist.map(tt);
        System.out.println("New list after mapping with TimesTwoFun(): " +  mylist2.toString());
    
    }

}
\end{lstlisting}  		

\end{enumerate}
 
10.1

\noindent
%%%%%%%%%%%%%%%%%%%%%%%%%% QUESTION 1 QUESTION 1 %%%%%%%%%%%%%%%%%%%%%%%%%%%%%%%%%%%%%

\begin{enumerate}[label=(\alph*),align=left, wide, labelwidth=!, labelindent=0pt]

%% 5,1a
  		\item Use the Adapter pattern to design a generic queue class called $LQueue<E>$ that implements interface $MyQueue<E>$ and that uses a $LinkedList<E>$ object.  Write the UML class diagram for the pattern.
  		
\begin{minipage}[h]{\linewidth}
			\raggedright
			\adjustbox{valign=t}{%
				\includegraphics[width=1\linewidth]{{UML}.png}
			}
\end{minipage}
  		
  		  		
%% 5,1b  		
  		\item Implement class $LQueue<E>$ in Java
  		\begin{lstlisting}[language = Java , frame = trBL , firstnumber = last , escapeinside={(*@}{@*)}]
class LQueue<E> implements MyQueue<E> {
    LinkedList<E> L = new LinkedList<E>();

    //Precondition: the queue is not empty.
    //Postcondition: the head of the queue is removed and returned
    //Returns: the element at the head of the queue is returned     
    public E head() {
        return L.getFirst();
    }
    
    //Precondition: the queue is not empty.
    //Postcondition: the head of the queue is removed and returned
    //Description: removes an element E from front of LinkedList interfacing as MyQueue.
    public E dequeue() {
            return L.removeFirst();
    }
    
    //Postcondition: the value is added to the tail of the structure
    //Paramaters: @e - the value to be added
    public void enqueue(E e) {
        L.addLast(e);
    }

    //Postcondition: the value is added to the tail of the structure
    //Paramaters: @e - the value to be added
    public int size() {
        return L.size();
    }

    //Postcondition: returns true if and only if the queue is empty
    //Returns: True iff the queue is empty.
    public boolean isEmpty() {
        return L.isEmpty();
    }

    //Precondition: a collection is not empty
    //Postcondition: all of the values in collection are added to queue
    public void addAll(Collection<? extends E> c) {
        Iterator<E> iter = (Iterator<E>) c.iterator();
        
        // get contents of collection and add to queue
        while(iter.hasNext()){
            enqueue(iter.next());
        }
    }
    
}
		\end{lstlisting}  		
  		
  		
%% 5,1b  	  		
\newpage
  		\item Write a class QueueTest
  		
  		\begin{minipage}[h]{\linewidth}
			\raggedright
			\adjustbox{valign=t}{%
				\includegraphics[width=1\linewidth]{{testing2}.png}
			}
	\end{minipage}
 
 \begin{lstlisting}[language = Java , frame = trBL , firstnumber = last , escapeinside={(*@}{@*)}]
 public class QueueTest {
    public static void main(String[] args)
    {
        LQueue<Integer> Q = new LQueue<>();
        Collection<Integer> list = new LinkedList<Integer>();
        
        // adding to collection
        list.add(2);
        list.add(4);
        list.add(3);
        list.add(5);
        list.add(11);
        list.add(8);

        // starting testing 
        System.out.println();
        System.out.println("TESTING head() && enqueue() ...");
        System.out.println("Adding 6 to queue");
        Q.enqueue(6);
        System.out.println("Checking if head of queue is 6 ... ");
        if(Q.head() == 6)
            System.out.println("   PASS");
        else
            System.out.println("   FAIL");
        
        
        System.out.println();
        System.out.println("Size of Queue is " + Q.size());
        System.out.println("Is queue empty? " + Q.isEmpty());
        
        System.out.println();
        System.out.println("TESTING dequeue() ...");
        Q.dequeue();
        System.out.println("Is queue empty? " + Q.isEmpty());
        
        System.out.println();
        System.out.println("Adding ALL collection to queue: 2, 4, 5, 11, 8");
        Q.addAll(list);
        System.out.println("Que contents:");
        while(! Q.isEmpty()){
            System.out.print(Q.head() + " " );
            Q.dequeue();
        }
        System.out.println();
    }
 
}
 \end{lstlisting}
  		
\end{enumerate}
 
 \newpage
10.2
 
 \begin{enumerate}[label=(\alph*),align=left, wide, labelwidth=!, labelindent=0pt]

%% 5,2a
\item Use the Singleton patternf or a new Java class Stdout that has one instance.  A programmer can use the instance to print txt lines (String objects) to the terminal with the method printlnine();

 \begin{lstlisting}[language = Java , frame = trBL , firstnumber = last , escapeinside={(*@}{@*)}]
 // Singleton class to print only one message per instance
class Stdout {
    
    //Postcondition: line is set by string and displaced, any other instance 
    //              will have the same line
    
    public void printline(String s)
    {
        line = s;
        System.out.println(s);        
    }
    
    //used to display as singleton
    public void printline()
    {
        System.out.println(line);
    }
    public static Stdout getInstance(){
        return stdout;
    }

    public String line;
    private Stdout(){}  // private constructor
    private static Stdout stdout = new Stdout();
    // static variable for singleton instance
    private static Stdout Stdout_instance = null;
    
}
 \end{lstlisting}
 \newpage
  		
  		\begin{minipage}[h]{\linewidth}
			\raggedright
			\adjustbox{valign=t}{%
				\includegraphics[width=1\linewidth]{{testing3}.png}
			}
	\end{minipage}
	
\begin{lstlisting}[language = Java , frame = trBL , firstnumber = last , escapeinside={(*@}{@*)}]
public class PrintTest {
    public static void main(String[] args){
        Stdout out = Stdout.getInstance();
        out.printline("Testing 1..5..6");
        Stdout out1 = Stdout.getInstance();
        out1.printline();
    }
}
 \end{lstlisting}


\end{enumerate}

\newpage
7.2
 
 \begin{enumerate}[label=(\alph*),align=left, wide, labelwidth=!, labelindent=0pt]

%% 5,2a
\item Implement a generic class $Pair<K,V>$ that stores pairs of (key, value) pairs.

 \begin{lstlisting}[language = Java , frame = trBL , firstnumber = last , escapeinside={(*@}{@*)}]
class Pair<K,V> implements Cloneable {   

    //Constructor for Pair
    //Parameters: K ~ key, V ~ value
    public Pair(K k, V v)
    {
        this.key = k;
        this.value = v;
    }
        
    //helper function to get key    
    public K k(){
        return key;
    }
    
    //helper function to get value
    public V v(){
        return value;
    }
    
    @Override
    public boolean equals(Object obj){
        if(this.toString().matches(obj.toString()))
            return true;
        else
            return false;
    }

    @Override
    public int hashCode() {
        int hash = 7;
        hash = 73 * hash + Objects.hashCode(this.key);
        hash = 73 * hash + Objects.hashCode(this.value);
        return hash;
    }
    
    @Override
    public String toString(){
        return this.k().toString() + ", " + v();
    }
    
    @Override
    public Object clone() throws CloneNotSupportedException{
        return super.clone();
    }
            
    public K key;
    public V value;
}
}
 \end{lstlisting}
 
 \newpage
 \item Write a PairTest class to test all functions 
  		
  		\begin{minipage}[h]{\linewidth}
			\raggedright
			\adjustbox{valign=t}{%
				\includegraphics[width=1\linewidth]{{testing4}.png}
			}
	\end{minipage}
	
	\newpage
	
\begin{lstlisting}[language = Java , frame = trBL , firstnumber = last , escapeinside={(*@}{@*)}]
public class PairTest {
    
    
    static public void serializeTest(Pair pair) throws IOException{
        Writer fileWriter = new FileWriter("Pair.txt");
        System.out.println("Pair being writen to file -- Pair.txt");
        System.out.println("   Key = " + pair.k());
        System.out.println("   Value = " + pair.v());
        try (PrintWriter out = new PrintWriter(fileWriter)) {
            out.print(pair.toString());
        }
        
        File file = new File("Pair.txt"); 
        BufferedReader br = new BufferedReader(new FileReader(file));
  
        String line;
        System.out.println("Pair being read from file -- Pair.txt");

        while ((line = br.readLine()) != null) {
            String[] split = line.split(",");
            System.out.println("   Key = " + split[0]);
            System.out.println("   Value = " + split[1].trim());
        }
        
        System.out.println("Testing done");
            
    }
    
    public static void main(String[] args)
    {
        System.out.println("PAIRS DEFINED");
        Pair<Integer,String> p = new Pair<>(1,"One");
        Pair<Integer,String> p1 = new Pair<>(1,"One");
        Pair<Integer,String> p2 = new Pair<>(2,"Two");
        
        System.out.println("p = [" + p.toString() + "]");
        System.out.println("p1 = [" + p.toString() + "] ~exactly equal to p~");
        System.out.println("p2 = [" + p.toString() + "]");

        
        // test equality
        System.out.println("");
        System.out.println("TESTING EQUALITY");
        System.out.println("Does [" + p +"] equal [" + p1 +"] ?");
        System.out.println("Result: " + p.equals(p1));
        System.out.println("Does [" + p +"] equal [" + p2 +"] ?");
        System.out.println("Result: " + p.equals(p2));
        
        // test clone
        System.out.println("");
        System.out.println("TESTING clone()");
        System.out.println("Cloning p as p3");
        try{
            Pair p3 = (Pair<Integer,String>)p.clone();
            System.out.println("Result: p3 = [" + p3.toString() + "]");
        }catch(CloneNotSupportedException c){
            System.out.println("Cloning failed");
        }  
        
        
        // test serialization
        System.out.println("");
        System.out.println("TESTING SERIALIZATION");
        try {
            serializeTest(p);
        } catch (IOException ex) {
            System.out.println("");
        }
        
        System.out.println();
        System.out.println("TESTING hashCode()");
        System.out.println("hashCode for p and p1 (identical Pairs) = " + p.hashCode()+ " and " + p1.hashCode());
        System.out.println("hashCode for p2 = " + p2.hashCode());
    }
}
 \end{lstlisting}


\end{enumerate}
  		

  	